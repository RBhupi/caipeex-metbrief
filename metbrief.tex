%%%%%%%%%%%%%%%%%%%%%%%%%%%%%%%%%%%%%%%%%
% Short Three-Column Newsletter
% LaTeX Template
% Version 1.0 (11/9/13)
%
% Original author:
% Frits Wenneker (http://www.howtotex.com) 
% With extensive modifications by:
% Vel (vel@latextemplates.com)
% 
% This template has been downloaded from:
% http://www.LaTeXTemplates.com
%
% License:
% CC BY-NC-SA 3.0 (http://creativecommons.org/licenses/by-nc-sa/3.0/)
%
%%%%%%%%%%%%%%%%%%%%%%%%%%%%%%%%%%%%%%%%%

%----------------------------------------------------------------------------------------
%	PACKAGES AND DOCUMENT CONFIGURATIONS
%----------------------------------------------------------------------------------------

\documentclass[12pt,a4paper]{article} % Paper type (a4paper, usletter or legal) and font size (10, 11 or 12)

\usepackage{graphicx} % Required for including images
\usepackage{amssymb,amsmath} % Math packages
\usepackage{multicol} % Required for the three-column layout of the document
\usepackage{url} % Clickable links
\usepackage[hidelinks]{hyperref} % Link colors and PDF behavior in Acrobat
\usepackage{fancyhdr} % Required to define custom headers/footers
\pagestyle{fancy} % Enables the custom headers/footers for all pages following this
\usepackage{float}
\usepackage[margin=0.5in, landscape]{geometry}


\fancyfoot[C]{\thepage}                                                                                                                                                                                                                                           

%-----------------------------------------------------------
% Header and footer
\lfoot{\footnotesize % Left footer containing newsletter contact information
CAIPEEX Met. brief. for \today\\
\href{http://www.tropmet.res.in/}{tropmet.res.in/}
}

\cfoot{} % Empty center footer

\rfoot{\footnotesize ~\\ Page \thepage} % Right footer - page counter
\fancyhead{}

\renewcommand{\headrulewidth}{0.0pt} % No horizontal rule for the header
\renewcommand{\footrulewidth}{0.4pt} % Horizontal rule separating the footer from the document
%-----------------------------------------------------------

%-----------------------------------------------------------
% Define separators
\newcommand{\HorRule}[1]{\noindent\rule{\linewidth}{#1}} % Creates a horizontal rule
\newcommand{\SepRule}{\noindent	% Creates a shorter separator rule
\begin{center}
\rule{250pt}{1pt} % Page width and rule width
\end{center}
}
%-----------------------------------------------------------

%-----------------------------------------------------------

%----------------------------------------------------------------------------------------
%	TITLE
%----------------------------------------------------------------------------------------
\title{CAIPEEX Phase IV: Meteorological Briefing}% from Pune} % Newsletter title
\begin{document}
\maketitle

\noindent\HorRule{3pt} \\[-0.75\baselineskip] % Thick horizontal rule
\HorRule{1pt} % Thin horizontal rule

%----------------------------------------------------------------------------------------
%	MAIN NEWS ITEM
%----------------------------------------------------------------------------------------

%\vspace{0.5cm}
%\SepRule
%\vspace{-0.5cm}

%\NewsItem{Highlights} % Main next item title
%\vspace{3pt} % Some extra whitespace since there is no author as for the news in the body of the newsletter

%{\itshape
%\begin{enumerate}
%\input{./text/highlights.tex}
%\end{enumerate}


%}
\begin{multicols}{3} % Begin the three-column layout
\tableofcontents
\SepRule % Small horizontal rule after the main news item
\vspace{0.3cm}

%\setlength{\columnsep}{16pt} % Uncomment to manually change the white space between columns


%----------------------------------------------------------------------------------------
%	OTHER NEWS
%----------------------------------------------------------------------------------------

\vfill
\section{Monsoon Features based on 05:30 IST Analysis}

\begin{itemize}
\item  Monsoon trough, continues to remain active with 2 embedded cyclonic circulations, (i) over northern parts of Haryana and neighbourhood (ii) over East Uttar Pradesh and adjoining Bihar.
\item A low pressure area has formed over northwest Bay of Bengal and neighbourhood and associated cyclonic circulation extends up to 4.5 km above mean sea level.
\item As the cross equatorial flow is likely to continue to remain weak, the rainfall activity could remain subdued over Peninsular India during next 4 days.
\item The cyclonic circulation at 3.1 km above mean sea level over South Interior Karnataka \& neighbourhood persists.
\end{itemize}

\end{multicols}

\begin{figure}[H]
\centering
\includegraphics[width=3in]{fig/onset_SW.png}
\includegraphics[width=5in, height=6in]{fig/seasonal-rain}
\includegraphics[width=5in, height=6in]{fig/onset_SW}
\caption{Seasonal rainfall and Monsoon onset dates (normal and actual).}
\end{figure}




%\begin{figure}[H]
%\centering
%\includegraphics[width=3in, height=4in]{fig/all-india.png}
%\includegraphics[width=3in, height=4in]{fig/homo.png}
%\caption{All India  (Left) and Homogeneous rainfall (Right)}
%\end{figure}



\section{Satellite Inference}
\begin{figure}[H]
\centering
\includegraphics[width=5in]{fig/3Dasiasec_ctbt}
\includegraphics[width=5in, height=6in]{fig/mslp_avg_spr-24}
%\includegraphics[width=3in]{fig/3Dtpw_a1}
\caption{IR cloud top temperatures and MSLP analysis.}
\end{figure}

%\begin{figure}[H]
%\centering
%\includegraphics[width=3in, trim=8cm 1cm 3cm 1cm, clip]{fig/3D_shear.jpg}
%\includegraphics[width=3in, trim=8cm 8cm 3cm 1cm, clip]{fig/3D_midsh.jpg}

%\caption{Mid-level shear (Left) and mid-level moisture from Rapid (Right).}
%\end{figure}


%\begin{figure}[H]
%\centering
%\includegraphics[width=3in, trim=8cm 1cm 3cm 1cm, clip]{fig/3D_lowcon.jpg}
%\includegraphics[width=3in, trim=8cm 1cm 3cm 1cm, clip]{fig/3D_updiv.jpg}

%\caption{Low level convergence (Left), Upeer level divergence  (Right).}
%\end{figure}

\section{Forecast}

%\subsection*{NUCM Temperature Forecast}
%\begin{figure}[H]
%\centering
%\includegraphics[width=3in]{fig/ncum_fcst_Tmax_day-1.png}
%\includegraphics[width=3in]{fig/ncum_fcst_Tmin_day-1.png}
%\caption{Maximum and minimum temperature}
%\end{figure}


\subsection*{GFS Wind Forecast}

\begin{figure}[H]
\centering
\includegraphics[width=2.5in]{fig/con-spr-wind_zonal850-0.png}
\includegraphics[width=2.5in]{fig/con-spr-wind_zonal200-0.png}
\caption{GFS Forecast.}
\end{figure}


\subsection*{IMD WRF Forecast (NA)}

\begin{figure}[H]
\centering
%\includegraphics[width=2.5in, height=2.5in, trim=20cm 5cm 10cm 4cm, clip]{fig/WRFd02-03-RAIN_06.png}
\includegraphics[width=5in, height=6in, trim=20cm 5cm 10cm 4cm, clip]{fig/WRFd02-ZWd850_06.png}
\includegraphics[width=5in, height=6in, trim=20cm 5cm 10cm 4cm, clip]{fig/WRFd02-ZWd500_06.png}
%\includegraphics[width=2.5in, height=2.5in, trim=20cm 5cm 10cm 4cm, clip]{fig/WRFd02-ZWd200_06.png}
\caption{Geopotential height, wind and isotach at 850 hPa, 500 hPa.}
\end{figure}







%\subsection*{IMD Thermodynamics Forecast (CAPE, CINE, LCLH)}% (Observations and Forecast) }
%\begin{figure}[H]
%\centering
%\fbox{\includegraphics[width=5in, trim=15cm 5cm 8cm 10cm, clip]{fig/WRFd01-CAPE_06.png}} \hskip 0.1cm
%
%\fbox{\includegraphics[width=5in, trim=15cm 5cm 8cm 10cm, clip]{fig/WRFd01-CIN_06.png}}\\
%\vskip 0.1cm
%\fbox{\includegraphics[width=5in, trim=15cm 5cm 8cm 10cm, clip]{fig/WRFd01-LCL_06.png}}
%\caption{IMD Forecast products CAPE (left), CINE (right) and  LCLH (bottom)}
%\end{figure}


%\begin{figure}[H]
%\centering
%\includegraphics[width=4in]{fig/SOLAPUR.png}\\
%\caption{T-Phi grams derived from INSAT-3D Sounder for Solapur when clear sky pixels available.}
%\end{figure}


\subsection*{IMD Meteogram}
\begin{figure}[H]
\centering
%trim=8cm 1cm 10cm 5cm, clip
\includegraphics[width=3in, height=6in, trim=0 0 25cm 0, clip]{fig/SLP-meteogram}
%\includegraphics[width=6in, height=6in]{fig/SLP-meteogram}
\caption{Meteogram for Solapur}
\end{figure}


\section{Trajectories}
%\begin{figure}[H]
%\centering
%trim=8cm 1cm 10cm 5cm, clip
%\includegraphics[width=3in]{fig/concwinds925-9.png}
%\includegraphics[width=3in]{fig/aod10days-9.png}
%\caption{Dust Forecast (Left) and Aerosol  Optical Depth (Right)}
%\end{figure}

\begin{figure}[H]
\centering
%[width=4in, height=6in, trim=2cm 3cm 3cm 5cm, clip]
\includegraphics[width=4in, height=5in]{manual_fig/backward}
\includegraphics[width=4in, height=5in]{manual_fig/forward}
\caption{84 hours Trajectories.}
\end{figure}



%\begin{figure}[H]
%\centering
%\includegraphics[width=6in, trim=0cm 3cm 0cm 0cm, clip]{fig/soilm_7days.png}
%\caption{Soil moisture}
%\end{figure}



\section{WRF Mesoscale Forecast}
\subsection*{Skew-T Plots}
\begin{figure}[H]
\centering
%trim=8cm 1cm 10cm 5cm, clip

\includegraphics[width=5in, height=6in]{{fig/Solapur.000013}.png}
\includegraphics[width=5in, height=6in]{{fig/Solapur.000015}.png}
\caption{Skewt-T plots for Solapur.}
\end{figure}


\subsection*{CAPE and CINE}
\begin{figure}[H]
\centering
%trim=8cm 1cm 10cm 5cm, clip

\includegraphics[width=5in, height=5in, trim=0cm 9cm 6cm 0cm, clip]{{fig/cape.000015}.png}
\includegraphics[width=5in, height=5in, trim=0cm 9cm 6cm 0cm, clip]{{fig/cine.000015}.png}
\caption{CAPE and CINE at 06 UTC }
\end{figure}




\subsection*{Max dBZ and Other Moisture Parameters}

\begin{figure}[H]
\centering
%trim=8cm 1cm 10cm 5cm, clip
\includegraphics[width=5in, height=5in, trim=0cm 9cm 6cm 0cm, clip]{{fig/mdbz.000015}.png}
\includegraphics[width=5in, height=5in, trim=0cm 9cm 6cm 0cm, clip]{{fig/mdbz.000017}.png}
\caption{Max dBZ at 06 UTC and 08 UTC.}
\end{figure}

\begin{figure}[H]
\centering
%trim=8cm 1cm 10cm 5cm, clip
\includegraphics[width=8in, height=7in]{{fig/lwp_iwp.000014}.png}
\caption{LWP and IWL at 07 UTC.}
\end{figure}

\begin{figure}[H]
\centering
%trim=8cm 1cm 10cm 5cm, clip
\includegraphics[width=6in, height=6in]{{fig/MR.000017}.png}
\caption{Mixing Ratio at 10 UTC for 1, 2, 3 and 4 Km levels.}
\end{figure}


\begin{figure}[H]
\centering
%trim=8cm 1cm 10cm 5cm, clip
\includegraphics[width=4in, height=4in, trim=0cm 9cm 6cm 0cm, clip]{{fig/wld.000015}.png}
\includegraphics[width=4in, height=4in, trim=0cm 9cm 6cm 0cm, clip]{{fig/wld.000017}.png}
\caption{Warm layer depth at 06 UTC and 08 UTC.}
\end{figure}



%\subsection*{Precipitation Tendency and LCL}

%\begin{figure}[H]
%\centering
%trim=8cm 1cm 10cm 5cm, clip
%\includegraphics[width=3in, height=3in, trim=0cm 9cm 6cm 0cm, clip]{{fig/precipt.000019}.png}
%\includegraphics[width=3in, height=3in, trim=0cm 9cm 6cm 0cm, clip]{{fig/precipt.000021}.png}
%\caption{Precip tendency at 06 UTC 08 UTC}
%\end{figure}

\begin{figure}[H]
\centering
%trim=8cm 1cm 10cm 5cm, clip
\includegraphics[width=5in, height=5in, trim=0cm 9cm 6cm 0cm, clip]{{fig/lcl.000015}.png}
\includegraphics[width=5in, height=5in, trim=0cm 9cm 6cm 0cm, clip]{{fig/lcl.000017}.png}
\caption{ LCL at 06 UTC 08 UTC}
\end{figure}


\section{Solapur Radar}
%\subsection*{Radar}
\begin{figure}[H]
\centering
%trim=8cm 1cm 10cm 5cm, clip
\includegraphics[width=3in, height=2.0in]{manual_fig/radarppi.jpeg}
\caption{Radar images Max dBZ.}
\end{figure}



\section{Rain Gauge Network}
\begin{figure}[H]
\centering
\includegraphics[width=6in, height=6in]{fig_obs/rain-gauge_output}
\caption{24-hour accumulation shown with bubble size and color (top), 24Hr cumulative rainfall (bottom).}

\end{figure}

%\section{Satellite Pass}

%\begin{figure}[H]
%\centering
%trim=8cm 1cm 10cm 5cm, clip
%\includegraphics[width=3in, height=2.5in]{manual_fig/sat_pass}
%\includegraphics[width=3in, height=2.5in]{manual_fig/sat_pass_table}
%\caption{Terra satellite pass over our site 05:40 UTC.}
%\end{figure}



%\newpage  % This will put the outlook on the new page.
\subsection*{Convective potential}
IR Satellite image and Radar shows no convective activity near our site. IITM WRF forecast shows dry conditions and strong inversion in the afternoon over Solapur.



\subsection*{Outlook}
Clear sky conditions with weak cumuliform clouds.\\
%There is NO satellite pass over our site today.
\subsection*{Decision}
It is suggested to take aerosol flight and watch for suitable seeding conditions.\\

\vfill % this will push the next line to the bottom of the page. Remove this if you dont like it. 
\textit{Contributors: Bhupendra Raut.} 




%----------------------------------------------------------------------------------------

\end{document} 
