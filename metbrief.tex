%%%%%%%%%%%%%%%%%%%%%%%%%%%%%%%%%%%%%%%%%
% Short Three-Column Newsletter
% LaTeX Template
% Version 1.0 (11/9/13)
%
% Original author:
% Frits Wenneker (http://www.howtotex.com) 
% With extensive modifications by:
% Vel (vel@latextemplates.com)
% 
% This template has been downloaded from:
% http://www.LaTeXTemplates.com
%
% License:
% CC BY-NC-SA 3.0 (http://creativecommons.org/licenses/by-nc-sa/3.0/)
%
%%%%%%%%%%%%%%%%%%%%%%%%%%%%%%%%%%%%%%%%%

%----------------------------------------------------------------------------------------
%	PACKAGES AND DOCUMENT CONFIGURATIONS
%----------------------------------------------------------------------------------------

\documentclass[12pt,a4paper]{article} % Paper type (a4paper, usletter or legal) and font size (10, 11 or 12)

\usepackage{graphicx} % Required for including images
\usepackage{amssymb,amsmath} % Math packages
\usepackage{multicol} % Required for the three-column layout of the document
\usepackage{url} % Clickable links
\usepackage[hidelinks, colorlinks = true]{hyperref} % Link colors and PDF behavior in Acrobat
\usepackage{fancyhdr} % Required to define custom headers/footers
\pagestyle{fancy} % Enables the custom headers/footers for all pages following this
\usepackage{float}
\usepackage[margin=0.5in, landscape]{geometry}


\fancyfoot[C]{\thepage}                                                                                                                                                                                                                                           

%-----------------------------------------------------------
% Header and footer
\lfoot{\footnotesize % Left footer containing newsletter contact information
CAIPEEX Met. brief. for \today | contributors Bhupendra Raut \hfill Page \thepage\\
%\href{http://www.tropmet.res.in/}{tropmet.res.in/}
}

\cfoot{} % Empty center footer
\rfoot{}
%\rfoot{\footnotesize ~\\ Page \thepage} % Right footer - page counter
\fancyhead{}

\renewcommand{\headrulewidth}{0.0pt} % No horizontal rule for the header
\renewcommand{\footrulewidth}{0.4pt} % Horizontal rule separating the footer from the document
%-----------------------------------------------------------

%-----------------------------------------------------------
% Define separators
\newcommand{\HorRule}[1]{\noindent\rule{\linewidth}{#1}} % Creates a horizontal rule
\newcommand{\SepRule}{\noindent	% Creates a shorter separator rule
\begin{center}
\rule{250pt}{1pt} % Page width and rule width
\end{center}
}
%-----------------------------------------------------------

%-----------------------------------------------------------

%----------------------------------------------------------------------------------------
%	TITLE
%----------------------------------------------------------------------------------------
\title{CAIPEEX Phase IV: Meteorological Briefing}% from Pune} % Newsletter title
\begin{document}
\maketitle

\noindent\HorRule{3pt} \\[-0.75\baselineskip] % Thick horizontal rule
\HorRule{1pt} % Thin horizontal rule

%----------------------------------------------------------------------------------------
%	MAIN NEWS ITEM
%----------------------------------------------------------------------------------------

%\vspace{0.5cm}
%\SepRule
%\vspace{-0.5cm}

%\NewsItem{Highlights} % Main next item title
%\vspace{3pt} % Some extra whitespace since there is no author as for the news in the body of the newsletter

%{\itshape
%\begin{enumerate}
%\input{./text/highlights.tex}
%\end{enumerate}


%}
\begin{multicols}{3} % Begin the three-column layout
\includegraphics[width=3.2in, height=3.8in]{fig/onset_SW.png}
\SepRule % Small horizontal rule after the main news item
\vspace{0.3cm}

%\setlength{\columnsep}{16pt} % Uncomment to manually change the white space between columns


%----------------------------------------------------------------------------------------
%	OTHER NEWS
%----------------------------------------------------------------------------------------

\vfill
\section{Monsoon Features based on 05:30 IST Analysis}

\begin{itemize}
\item  Monsoon trough, continues to remain active with 2 embedded cyclonic circulations, (i) over northern parts of Haryana and neighbourhood (ii) over East Uttar Pradesh and adjoining Bihar.
\item A low pressure area has formed over northwest Bay of Bengal and neighbourhood and associated cyclonic circulation extends up to 4.5 km above mean sea level.
\item As the cross equatorial flow is likely to continue to remain weak, the rainfall activity could remain subdued over Peninsular India during next 4 days.
\item The cyclonic circulation at 3.1 km above mean sea level over South Interior Karnataka \& neighbourhood persists.
\end{itemize}
\end{multicols}




\section{Synoptic Charts}
%\subsection*{GFS Wind Forecast}
\begin{figure}[H]
\centering
\includegraphics[width=3.8in, height=3in]{fig/06hgfs_mslp}
\includegraphics[width=3.8in, height=3in]{fig/06hgfs_850wind}\\
\includegraphics[width=3.8in, height=3in]{fig/06hgfs_500wind}
\includegraphics[width=3.8in, height=3in]{fig/06hgfs_200wind}
\caption{GFS wind forecast at 850, 500 and 200 hPa levels.}
\end{figure}





\section{Satellite Inference}
\begin{figure}[H]
\centering
\includegraphics[width=4in, height=3.2in, page=1, trim=1cm 10cm 1cm 6cm, clip]{satbltn.pdf}
\includegraphics[width=4in, height=3.2in]{fig/3Dasiasec_ctbt}\\
\includegraphics[height=3in, trim=10cm 4cm 6cm 4cm, clip]{fig/3Dhum_prof620_a1}
\caption{Salient Features in IR, IR CTBT and Mid-level moisture from rapid. \href{http://satellite.imd.gov.in/img/3Dswsec_swir.jpg}{Click here for latest HiRes SWIR image.}}
\end{figure}



%\begin{figure}[H]
%\centering
%\includegraphics[width=4in]{fig/SOLAPUR.png}\\
%\caption{T-Phi grams derived from INSAT-3D Sounder for Solapur when clear sky pixels available.}
%\end{figure}



\section{Dust and HYSPLIT Trajectories}

\begin{figure}[H]
\centering
%[width=4in, height=6in, trim=2cm 3cm 3cm 5cm, clip]
\includegraphics[width=3in, trim=5cm 0cm 6cm 0cm, clip]{fig/dust-10}
\includegraphics[height=3in, trim=2cm 3cm 3cm 5cm, clip]{manual_fig/backward}
\includegraphics[height=3in, trim=2cm 3cm 3cm 5cm, clip]{manual_fig/forward}
\caption{Dust Forecast and 48 hours Trajectories.}
\end{figure}




\section{IMD Meteogram}
\begin{figure}[H]
\centering
%[width=3in, height=6in, trim=0 0 25cm 0, clip]
\includegraphics[width=6in]{fig/SLP-meteogram}
\caption{Meteogram for Solapur}
\end{figure}




\section{IITM WRF Mesoscale Forecast}
\subsection*{Skew-T Plots}
\begin{figure}[H]
\centering
%trim=8cm 1cm 10cm 5cm, clip

\includegraphics[width=5in, height=6in]{{fig/Solapur.000013}.png}
\includegraphics[width=5in, height=6in]{{fig/Solapur.000015}.png}
\caption{Skewt-T plots for Solapur.}
\end{figure}


\subsection*{CAPE and Max dBZ}
\begin{figure}[H]
\centering
%trim=8cm 1cm 10cm 5cm, clip
\includegraphics[height=2.7in, trim=0cm 9cm 6cm 0cm, clip]{{fig/cape.000015}.png}
\includegraphics[height=2.7in, trim=0cm 9cm 6cm 0cm, clip]{{fig/cape.000017}.png}
\caption{CAPE at 06 and 08 UTC }
\end{figure}


\begin{figure}[H]
\centering
%trim=8cm 1cm 10cm 5cm, clip
\includegraphics[width=4in, height=3in, trim=0cm 9cm 6cm 0cm, clip]{{fig/mdbz.000015}.png}
\includegraphics[width=4in, height=3in, trim=0cm 9cm 6cm 0cm, clip]{{fig/mdbz.000017}.png}
\caption{Max dBZ at 06 UTC and 08 UTC.}
\end{figure}



\subsection*{Moisture Parameters}

\begin{figure}[H]
\centering
%trim=8cm 1cm 10cm 5cm, clip
\includegraphics[width=8in, trim=0in 10in 0in 10in, clip]{{fig/lwp_iwp.000014}.png}
\caption{LWP and IWL at 07 UTC.}
\end{figure}


\begin{figure}[H]
\centering
%trim=8cm 1cm 10cm 5cm, clip
\includegraphics[width=6in, height=6in]{{fig/MR.000017}.png}
\caption{Mixing Ratio at 10 UTC for 1, 2, 3 and 4 Km levels.}
\end{figure}




\subsection*{LCL and WLD}

\begin{figure}[H]
\centering
%trim=8cm 1cm 10cm 5cm, clip
\includegraphics[height=2.7in, trim=0cm 9cm 6cm 0cm, clip]{{fig/lcl.000015}.png}
\includegraphics[height=2.7in, trim=0cm 9cm 6cm 0cm, clip]{{fig/lcl.000017}.png}
\caption{ LCL at 08 UTC and 10 UTC}
\end{figure}


\begin{figure}[H]
\centering
%trim=0cm 9cm 6cm 0cm
\includegraphics[height=3in]{{fig/wld.000015}.png}
\includegraphics[height=3in]{{fig/wld.000017}.png}
\caption{Warm layer depth at 08 UTC and 10 UTC.}
\end{figure}



\section{Solapur Observations}

\subsection*{Rain Gauge Network}
\begin{figure}[H]
\centering
\includegraphics[width=6in, height=6in]{fig/rain-gauge_output}
\caption{24-hour accumulation shown with bubble size and color (top), 24Hr cumulative rainfall (bottom).}

\end{figure}


\subsection*{Radar}
\begin{figure}[H]
\centering
%trim=8cm 1cm 10cm 5cm, clip
\includegraphics[width=5in]{fig/caz_slp}
\includegraphics[width=5in]{fig/vp2_slp}
%\caption{Radar Reflectivity [dBZ] PPI .}
\end{figure}








%\newpage  % This will put the outlook on the new page.
\subsection*{Convective potential}
IR Satellite image and Radar shows no convective activity near our site. IITM WRF forecast shows dry conditions and strong inversion in the afternoon over Solapur.



\subsection*{Outlook}
Clear sky conditions with weak cumuliform clouds.\\
%There is NO satellite pass over our site today.
\subsection*{Decision}
It is suggested to take aerosol flight and watch for suitable seeding conditions.\\

\vfill % this will push the next line to the bottom of the page. Remove this if you dont like it. 
\textit{Contributors: Bhupendra Raut.} 




\section{Satellite Pass}

\begin{figure}[H]
\centering
%trim=8cm 1cm 10cm 5cm, clip
\includegraphics[width=6in]{manual_fig/sat_pass}
%\includegraphics[width=3in, height=2.5in]{manual_fig/sat_pass_table}
\caption{GPM satellite pass over our site 09:14 UTC.}
\end{figure}



%----------------------------------------------------------------------------------------

\end{document} 
