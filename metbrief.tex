%%%%%%%%%%%%%%%%%%%%%%%%%%%%%%%%%%%%%%%%%
% Short Three-Column Newsletter
% LaTeX Template
% Version 1.0 (11/9/13)
%
% Original author:
% Frits Wenneker (http://www.howtotex.com) 
% With extensive modifications by:
% Vel (vel@latextemplates.com)
% 
% This template has been downloaded from:
% http://www.LaTeXTemplates.com
%
% License:
% CC BY-NC-SA 3.0 (http://creativecommons.org/licenses/by-nc-sa/3.0/)
%
%%%%%%%%%%%%%%%%%%%%%%%%%%%%%%%%%%%%%%%%%

%----------------------------------------------------------------------------------------
%	PACKAGES AND DOCUMENT CONFIGURATIONS
%----------------------------------------------------------------------------------------

\documentclass[10pt,a4paper]{article} % Paper type (a4paper, usletter or legal) and font size (10, 11 or 12)

\setlength\topmargin{-48pt} % Top margin
\setlength\headheight{0pt} % Header height
\setlength\textwidth{7.0in} % Text width
\setlength\textheight{9.5in} % Text height
\setlength\oddsidemargin{-30pt} % Left margin
\setlength\evensidemargin{-30pt} % Left margin (even pages) - only relevant with 'twoside' article option

\usepackage{charter} % Charter font for main content

\frenchspacing % Reduces space after periods to make text more compact for a three-column layout

\usepackage{graphicx} % Required for including images
\usepackage{amssymb,amsmath} % Math packages
\usepackage{multicol} % Required for the three-column layout of the document
\usepackage{url} % Clickable links
\usepackage{enumitem} % Reduces the amount of space within and between lists with [noitemsep,nolistsep]
\usepackage{marvosym} % Required for the use of symbols
\usepackage{wrapfig} % Allows wrapping text around figures
\usepackage[T1]{fontenc} % Use 8-bit encoding that has 256 glyphs
\usepackage{datetime} % Required for defining a custom date style
\newdateformat{mydate}{\monthname[\THEMONTH] \THEYEAR} % Set a custom date format
\usepackage[colorlinks=false]{hyperref} % Link colors and PDF behavior in Acrobat
\usepackage{fancyhdr} % Required to define custom headers/footers
\pagestyle{fancy} % Enables the custom headers/footers for all pages following this
\usepackage{float}
%-----------------------------------------------------------
% Header and footer
\lfoot{\footnotesize % Left footer containing newsletter contact information
CAIPEEX Met. brief. for \today\\
\Mundus\ \href{http://www.tropmet.res.in/}{tropmet.res.in/} \quad
\Telefon\ (+91) 020 xxxx  \quad
\Letter\ \href{mailto:thara@tropmet.res.in}{thara@tropmet.res.in}
}

\cfoot{} % Empty center footer

\rfoot{\footnotesize ~\\ Page \thepage} % Right footer - page counter

\renewcommand{\headrulewidth}{0.0pt} % No horizontal rule for the header
\renewcommand{\footrulewidth}{0.4pt} % Horizontal rule separating the footer from the document
%-----------------------------------------------------------

%-----------------------------------------------------------
% Define separators
\newcommand{\HorRule}[1]{\noindent\rule{\linewidth}{#1}} % Creates a horizontal rule
\newcommand{\SepRule}{\noindent	% Creates a shorter separator rule
\begin{center}
\rule{250pt}{1pt} % Page width and rule width
\end{center}
}
%-----------------------------------------------------------

%-----------------------------------------------------------
% Define title and article styles
\newcommand{\NewsletterName}[1]{ % Newsletter title
\begin{center}
\Huge \usefont{T1}{fvs}{b}{n} % Use the Bera Sans Bold font
#1
\end{center}	
\par \normalsize \normalfont}

\newcommand{\JournalIssue}[1]{ % Date and issue number at the top of the newsletter
\hfill \textsc{\today} % Right-aligned date and issue number
\par \normalsize \normalfont}

\newcommand{\NewsItem}[1]{ % News item title
\usefont{T1}{fvs}{n}{n} % Use the Bera Sans Normal font
\vspace{24pt}\large #1\vspace{3pt} % Print the title with space around it in a larger font size
\par \normalsize \normalfont}

\newcommand{\NewsAuthor}[1]{ % Author name under the item title
\hfill by \textsc{#1} \vspace{20pt} % Right-aligned author name in small caps with space after it
\par \normalfont}		

%----------------------------------------------------------------------------------------
%	TITLE
%----------------------------------------------------------------------------------------

\begin{document}
\JournalIssue{4} % Issue number
\NewsletterName{\Large{CAIPEEX Phase IV:\\Meteorological briefing from Pune}} % Newsletter title

\noindent\HorRule{3pt} \\[-0.75\baselineskip] % Thick horizontal rule
\HorRule{1pt} % Thin horizontal rule

%----------------------------------------------------------------------------------------
%	MAIN NEWS ITEM
%----------------------------------------------------------------------------------------

\vspace{0.5cm}
\SepRule
\vspace{-0.5cm}

\NewsItem{Highlights} % Main next item title
\vspace{3pt} % Some extra whitespace since there is no author as for the news in the body of the newsletter

{\itshape
\begin{enumerate}
\item Southwest monsoon further advanced into remaining parts of Maharashtra.
\end{enumerate}


}

\tableofcontents

\vspace{0.5cm}
\SepRule % Small horizontal rule after the main news item
\vspace{0.5cm}

%\setlength{\columnsep}{16pt} % Uncomment to manually change the white space between columns


%----------------------------------------------------------------------------------------
%	OTHER NEWS
%----------------------------------------------------------------------------------------

\section{Advance of Southwest Monsoon}
\begin{multicols}{2} % Begin the three-column layout
\begin{itemize}
\item  Monsoon trough, continues to remain active with 2 embedded cyclonic circulations, (i) over northern parts of Haryana and neighbourhood (ii) over East Uttar Pradesh and adjoining Bihar.
\item A low pressure area has formed over northwest Bay of Bengal and neighbourhood and associated cyclonic circulation extends up to 4.5 km above mean sea level.
\item As the cross equatorial flow is likely to continue to remain weak, the rainfall activity could remain subdued over Peninsular India during next 4 days.
\item The cyclonic circulation at 3.1 km above mean sea level over South Interior Karnataka \& neighbourhood persists.
\end{itemize}


\begin{figure}[H]
\centering
%\includegraphics[width=3in]{fig/onset_SW.png}
\includegraphics[width=3in]{fig/seasonal-rain}
\caption{Seasonal rainfall for meteorological subdivisions since 1$^{st}$ June 2018.}
\end{figure}
\end{multicols}




\begin{figure}[H]
\centering
\includegraphics[width=3in, height=4in]{fig/all-india.png}
\includegraphics[width=3in, height=4in]{fig/homo.png}
\caption{All India  (Left) and Homogeneous rainfall (Right)}
\end{figure}


\section{Large-scale Features}


\subsection{Satellite View}

\begin{figure}[H]
\centering
\includegraphics[width=3in]{fig/3Dasiasec_vis.jpg}
\includegraphics[width=3in]{fig/3Dasiasec_ctbt.jpg}\\
%\includegraphics[width=3in]{fig/3Dtpw_a1}
\caption{Visible image (Left), IR cloud top temperatures (Right) and total Precipitable water (Bottom).}
\end{figure}

\begin{figure}[H]
\centering
%\includegraphics[width=3in, trim=8cm 1cm 3cm 1cm, clip]{fig/3D_shear.jpg}
\includegraphics[width=3in, trim=8cm 1cm 3cm 1cm, clip]{fig/3D_midsh.jpg}
\includegraphics[width=3in, trim=8cm 1cm 3cm 1cm, clip]{fig/3D_shten.jpg}
\caption{Mid-level shear (Left) and shear tendency (Right).}
\end{figure}


\begin{figure}[H]
\centering
\includegraphics[width=3in, trim=8cm 1cm 3cm 1cm, clip]{fig/3D_lowcon.jpg}
\includegraphics[width=3in, trim=8cm 1cm 3cm 1cm, clip]{fig/3D_updiv.jpg}

\caption{Low level convergence (Left), Upeer level divergence  (Right).}
\end{figure}

\section{Forecast}
\subsection{GFS Wind Forecast}

\begin{figure}[H]
\centering
\includegraphics[width=3in]{fig/con-spr-wind_zonal850-0.png}
\includegraphics[width=3in]{fig/con-spr-wind_zonal200-0.png}
\caption{GFS Forecast.}
\end{figure}




\subsection{NUCM Forecast}
\begin{figure}[H]
\centering
\includegraphics[width=3in]{fig/ncum_fcst_Tmax_day-1.png}
\includegraphics[width=3in]{fig/ncum_fcst_Tmin_day-1.png}
\caption{Maximum and minimum temperature}
\end{figure}


\begin{figure}[H]
\centering
\includegraphics[width=6in]{fig/ncum_fcst_Tmax_tend.png}\\
\vskip 0.5cm
\includegraphics[width=6in]{fig/ncum_fcst_Tmin_tend.png}
\caption{Maximum (Top) and minimum (bottom) 2m temperature Tendency}
\end{figure}



\subsection{IMD WRF forecast}

\begin{figure}[H]
\centering
\includegraphics[width=3in, trim=20cm 5cm 10cm 4cm, clip]{fig/WRFd02-03-RAIN_06.png}
\includegraphics[width=3in, trim=20cm 5cm 10cm 4cm, clip]{fig/WRFd02-ZWd850_06.png}
\includegraphics[width=3in, trim=20cm 5cm 10cm 4cm, clip]{fig/WRFd02-ZWd500_06.png}
\includegraphics[width=3in, trim=20cm 5cm 10cm 4cm, clip]{fig/WRFd02-ZWd200_06.png}
\caption{WRF Forecast.}
\end{figure}







\section{Thermodynamics (Observations and Forecast) }
\begin{figure}[H]
\centering
\fbox{\includegraphics[width=3in, trim=15cm 5cm 8cm 10cm, clip]{fig/WRFd01-CAPE_06.png}} \hskip 0.1cm
%
\fbox{\includegraphics[width=3in, trim=15cm 5cm 8cm 10cm, clip]{fig/WRFd01-CIN_06.png}}\\
\vskip 0.1cm
\fbox{\includegraphics[width=3in, trim=15cm 5cm 8cm 10cm, clip]{fig/WRFd01-LCL_06.png}}
\caption{IMD Forecast products CAPE (left), CINE (right) and  LCLH (bottom)}
\end{figure}


\begin{figure}[H]
\centering
\includegraphics[width=6in]{fig/SOLAPUR.png}\\
\caption{Solapur}
\end{figure}



\begin{figure}[H]
\centering
%trim=8cm 1cm 10cm 5cm, clip
\includegraphics[width=3in]{fig/concwinds925-10.png}
\includegraphics[width=3in]{fig/aod10days-10.png}
\caption{Dust Forecast (Left) and Aerosol  Optical Depth (Right)}
\end{figure}





\begin{figure}[H]
\centering
\includegraphics[width=6in, trim={0 3cm 0 0}]{fig/soilm_7days.png}
\caption{Soil moisture}
\end{figure}



\section{WRF Forecast}
\subsection{Skew-T Plots}

\begin{figure}[H]
\centering
%trim=8cm 1cm 10cm 5cm, clip
\includegraphics[width=3in]{{fig/Solapur.000010}.png}
\includegraphics[width=3in]{{fig/Tuljapur.000010}.png}
\caption{Skewt-T plots}
\end{figure}

\subsection{CAPE and CINE}
\begin{figure}[H]
\centering
%trim=8cm 1cm 10cm 5cm, clip
\includegraphics[width=3in, trim=0cm 9cm 6cm 0cm, clip]{{fig/cape.000013}.png}
\includegraphics[width=3in, trim=0cm 9cm 6cm 0cm, clip]{{fig/cine.000013}.png}
\caption{CAPE and CINE at 06 UTC}
\end{figure}


\subsection{Max dBZ and Warm Layer Depth}
\begin{figure}[H]
\centering
%trim=8cm 1cm 10cm 5cm, clip
\includegraphics[width=3in, trim=0cm 9cm 6cm 0cm, clip]{{fig/mdbz.000013}.png}
\includegraphics[width=3in, trim=0cm 9cm 6cm 0cm, clip]{{fig/wld.000013}.png}
\caption{CAPE and CINE at 06 UTC}
\end{figure}

\subsection{Precipitation Tendency and LCL}
\begin{figure}[H]
\centering
%trim=8cm 1cm 10cm 5cm, clip
\includegraphics[width=3in, trim=0cm 9cm 6cm 0cm, clip]{{fig/precipt.000012}.png}
\includegraphics[width=3in, trim=0cm 9cm 6cm 0cm, clip]{{fig/lcl.000013}.png}
\caption{Precip tendency and LCL at 06 UTC}
\end{figure}



%----------------------------------------------------------------------------------------

\end{document} 