\item Thunderstorm observed (from 1730 hours IST of yesterday to 0530 hours IST of today): at a few places over Vidarbha, Madhya Maharashtra, Marathwada, Coastal \& North Interior Karnataka, Konkan \& Goa and Andaman \& Nicobar Islands.
\item  The east-west shear zone roughly along Lat. 12°N between 3.1 \& 5.8 km above mean sea level across south Peninsula persists.
\item The cyclonic circulation extending upto 0.9 km above mean sea level over northwest Rajasthan and neighbourhood persists.
\item The cyclonic circulation extends upto 0.9 km above mean sea level over Bihar and adjoining Jharkhand persists.
\item The east-west trough from the cyclonic circulation over northwest Rajasthan to Bangladesh across Haryana, north Madhya Pradesh,
above cyclonic circulation over Bihar \& adjoining Jharkhand and Gangetic West Bengal extending upto 0.9 km above mean sea level
persists.
\item The cyclonic circulation over northwest Bay of Bengal and adjoining areas of West Bengal \& Bangladesh between 1.5 \& 2.1 km above
mean sea level persists.
\item The cyclonic circulation over westcentral Bay of Bengal \& neighbourhood between 3.1 \& 5.8 km above mean sea level persists.
\item The Western Disturbance as a cyclonic circulation at 3.1 km above mean sea level over east Afghanistan \& adjoining Pakistan persists.
\item It is very likely to
intensify further into a Severe Cyclonic Storm during next 24 hours. It is very likely to move nearly northwards and cross Gujarat coast
between Porbandar and Mahuva around Veraval \& Diu region as a Severe Cyclonic Storm with wind speed 110-120 kmph gusting to 135
kmph during early morning of 13th June.